\gdef\thisproblemorigin{180429, XI Кубок СПбГУ для школьников}
\gdef\thisproblemauthor{Иван Казменко}
\gdef\thisproblemdeveloper{Иван Казменко}
\gdef\messagelen{30}
\gdef\codelen{50}
\begin{problem}{Галактическая Служба Связи}
%{galactic-comm.in}{galactic-comm.out}
{\textsl{стандартный ввод}}{\textsl{стандартный вывод}}
{2 секунды}{256 мебибайт}{}

Анжела работает в Галактической Службе Связи.
Она прибыла на Марс, чтобы обеспечить связью обитателей
Фобоса и Деймоса, двух лун планеты.
Анжела уже запустила стандартные спутники-ретрансляторы на орбиты двух лун.
Осталось настроить их для взаимодействия.

Оборудование Галактической Службы Связи предназначено
для передачи сообщений между клиентами.
Стандартная форма одного сообщения "--- строка из $\messagelen$ двоичных цифр,
которая может принимать любое из $2^{\messagelen}$ различных значений.
Для передачи сообщения спутник-ретранслятор преобразует его в пакет "---
строку из $\codelen$ двоичных цифр.
Конкретные правила преобразования зависят от настроек спутника.
Далее пакет передаётся по каналу связи
на другой спутник-ретранслятор в месте назначения.
Наконец, этот второй спутник преобразует пакет обратно
в исходное сообщение из $\messagelen$ двоичных цифр и отсылает его клиенту.

В пакете не случайно больше двоичных цифр, чем в сообщении:
канал связи может частично испортить передаваемый пакет,
а сообщения нужно передавать без ошибок.
В этот раз оказалось, что в канале между Фобосом и Деймосом
наблюдается <<эффект залипающих единиц>>:
если на входе в канал в пакете есть ноль, левый сосед которого "--- единица,
то на выходе из канала этот ноль может сам превратиться в единицу.
Если таких нулей несколько, каждый из них может превратиться в единицу
независимо от остальных.
У первой из $\codelen$ цифр пакета нет левого соседа.
Например, если пересылаемый пакет начинается на <<\texttt{0100110}...>>,
третья и седьмая цифры могут превратиться в единицы, то есть при получении
начало пакета может выглядеть как <<\texttt{0100110}...>>,
<<\texttt{0110110}...>>, <<\texttt{0100111}...>> или <<\texttt{0110111}...>>.

Итак, Анжеле нужно настроить спутники.
Для этого следует выбрать, как преобразовывать сообщение из $\messagelen$ цифр
в пакет из $\codelen$ цифр, а также как преобразовывать пакет
обратно в исходное сообщение, учитывая, что при передаче пакет
может быть повреждён эффектом залипающих единиц.
Придумайте свой вариант преобразований
и напишите программу, которая их реализует.

\Interaction

В этой задаче на каждом тесте ваше решение будет запущено два раза.

При первом запуске заданные сообщения следует преобразовать в пакеты.
Первая строка входных данных состоит из слова <<\texttt{send}>>.
Во второй строке записано целое число $n$ "--- количество сообщений,
которые нужно передать ($1 \le n \le 10\,000$).
Каждая из следующих $n$ строк содержит одно сообщение "--- строку
из $\messagelen$ двоичных цифр.

Решение должно вывести $n$ строк "--- пакеты,
в которые преобразованы сообщения, в том порядке,
в котором сообщения следуют во входных данных.

После этого жюри формирует входные данные для второго запуска решения.
Для этого к каждому выведенному пакету применяется эффект залипающих единиц.
Кроме того, жюри может поменять пакеты местами произвольным образом.
В каждом тесте заранее зафиксировано,
как именно действует эффект залипающих единиц на каждый пакет,
а также как именно будут переставлены пакеты.

При втором запуске из полученных пакетов нужно восстановить исходные сообщения.
Первая строка входных данных состоит из слова <<\texttt{receive}>>.
Во второй строке записано целое число $n$ "--- количество пакетов,
которые нужно получить (число $n$ такое же, как при первом запуске).
Каждая из следующих $n$ строк содержит один пакет "--- строку
из $\codelen$ двоичных цифр.
Гарантируется, что будет дан именно тот набор пакетов,
который получился у жюри из ответа на первый запуск
после преобразований, описанных выше.

Решение должно вывести $n$ строк "--- исходные сообщения
в том порядке, в котором пакеты следуют во входных данных.
Напомним, что этот порядок не обязательно совпадает с исходным
порядком сообщений.

\Examples

\begin{examplewide}
\exmp{%
send
4
000000000000000000000000000000
010101010101010101010101010101
101010101010101010101010101010
111111111111111111111111111111
}{%
00000000000000000000000000000000000000000000000000
00111001001001000100010011010100011011101010111101
10101010101010101010101010101010101010101010101010
00000000000000000000000001111111111111111111111111
}%
\end{examplewide}

\begin{examplewide}
\exmp{%
receive
4
00000000000000000000000000000000000000000000000000
00111101101101100110011011111110011111111111111111
00000000000000000000000001111111111111111111111111
11111111101111111111101111101010101010111010101010
}{%
000000000000000000000000000000
010101010101010101010101010101
111111111111111111111111111111
101010101010101010101010101010
}%
\end{examplewide}

\Explanations

Выше показаны два запуска решения Анжелы на первом тесте.
Мы не знаем, как именно оно работает.

Между первым и вторым запуском некоторые пакеты
пострадали от эффекта залипающих единиц.
Первый и четвёртый пакеты не могли от него измениться.
Во втором пакете все нули, левыми соседями которых были единицы,
также стали единицами.
В третьем пакете эффект сработал лишь частично,
и в основном пострадала первая половина пакета.
Кроме того, перед вторым запуском
третий и четвёртый пакеты поменялись местами.

Несмотря на изменения в пакетах, решению Анжелы удалось
правильно восстановить исходные сообщения.

\end{problem}
