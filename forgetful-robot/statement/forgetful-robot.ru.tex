\gdef\thisproblemorigin{180429p, пробный тур}
\gdef\thisproblemauthor{Иван Казменко}
\gdef\thisproblemdeveloper{Иван Казменко}
\begin{problem}{Забывчивый робот-посыльный}
%{forgetful-robot.in}{forgetful-robot.out}
{\textsl{стандартный ввод}}{\textsl{стандартный вывод}}
{2 секунды}{256 мебибайт}{}

Алиса хочет передать Бобу целое число $a$ от $1$ до $10$.
Им помогает робот-посыльный Карл.

Алиса говорит Карлу какое-то целое число $x$ от $1$ до $1000$,
после чего Карл идёт к Бобу и должен повторить это число.
К сожалению, Карл забывчив и запоминает только примерное значение $x$.
Поэтому, когда посыльный придёт к Бобу, он может вспомнить число $x$ точно,
а может ошибиться на единицу и передать $x - 1$ или $x + 1$ вместо $x$.

Алиса и Боб хотят заранее договориться, как передавать числа,
чтобы Боб всегда мог восстановить Алисино число $a$
независимо от того, ошибётся Карл или нет.
Напишите программу, которая реализует такую договорённость.

\Interaction

В этой задаче ваше решение будет запущено два раза.

Первый запуск моделирует общение Алисы и Карла.
Единственная строка входных данных будет иметь вид
<<\texttt{Alice~$a$}>>, где $a$ "--- целое число от $1$ до $10$,
которое Алиса хочет передать Бобу.
Программа должна вывести $x$ "--- целое число $x$ от $1$ до $1000$,
которое Алиса должна сказать Карлу.
После этого необходимо завершить работу программы.

Второй запуск моделирует общение Карла и Боба.
Единственная строка входных данных будет иметь вид
<<\texttt{Carl~$y$}>>, где $y$ "--- целое число от $x - 1$ до $x + 1$,
которое Карл сказал Бобу.
Программа должна вывести $a$ "--- то число, которое Алиса
на самом деле хотела передать.

Обратите внимание: во время второго запуска ваше решение
получает только число $y$, ему не даются ни $a$, ни $x$.

В каждом тесте зафиксировано число $a$,
а также поведение Карла в зависимости от сказанного ему числа.

\Examples

\begin{example}
\exmp{%
Alice 3
}{%
392
}%
\exmp{%
Carl 392
}{%
3
}%
\exmp{%
Carl 391
}{%
3
}%
\exmp{%
Carl 393
}{%
3
}%
\end{example}

\Explanations

В решении, используемом в примере, Алиса и Боб условились так:
при $a = 3$ Алиса передаёт число $x = 392$,
а при $391 \le y \le 393$ Боб понимает, что $a$ было равно трём.
Первый пример показывает первый запуск такого решения при $a = 3$.

Следующие три примера соответствуют первым трём тестам в системе
и рассматривают второй запуск при $a = 3$ и различном поведении Карла.
В первом случае Карл передаёт число без изменений,
во втором "--- уменьшает его на единицу,
а в третьем "--- увеличивает на единицу.

Конечно, в вашем решении Алиса и Боб могли условиться иначе,
так что в общем случае решение не обязано выдавать именно такие ответы
во время отдельных запусков.

\end{problem}
