\gdef\thisproblemorigin{180429p, practice session}
\gdef\thisproblemauthor{Ivan Kazmenko}
\gdef\thisproblemdeveloper{Ivan Kazmenko}
\begin{problem}{Forgetful Robot}
%{forgetful-robot.in}{forgetful-robot.out}
{\textsl{standard input}}{\textsl{standard output}}
{2 seconds}{512 mebibytes}{}

Alice wants to send Bob an integer $a$ from $1$ to $10$.
Carl the Robot is helping them do the transfer.

Alice says to Carl some integer $x$ from $1$ to $1000$,
and then Carl goes to Bob and has to repeat that integer.
Unfortunately, Carl tends to forget stuff, so he remembers $x$
only approximately.
As a result, when the robot gets to Bob, he may remember the exact value
of $x$, but may also introduce an off-by one error, and give Bob
$x - 1$ or $x + 1$ instead of $x$.

Alice and Bob want to agree in advance how to transfer numbers
so that Bob will always be able to restore the exact value of Alice's
integer $a$, regardless of whether Carl introduces an error.
Write a program that will implement such an agreement.

\Interaction

In this problem, your solution will be run twice on each test.

The first run models communication between Alice and Carl.
The only line of input will have the form
``\texttt{Alice~$a$}'', where $a$ is an integer from $1$ to $10$
that Alice wants to give to Bob.
Your program has to print an integer $x$ from $1$ to $1000$:
the number that Alice tells to Carl.
After that, terminate your program gracefully.

The second run models communication between Carl and Bob.
The only line of input will have the form
``\texttt{Carl~$y$}'', where $y$ is an integer from $x - 1$ to $x + 1$:
the number that Carl tells to Bob.
Your program has to print an integer $a$: the number that Alice
was initially going to transfer.
After that, terminate your program gracefully.

Please note: during the second run, your solution only gets the integer $y$,
it knows neither $a$ nor $x$.

In each test, the number $a$ is fixed in advance,
as well as Carl's behavior depending on the number given to him.

\Examples

\begin{example}
\exmp{%
Alice 3
}{%
392
}%
\exmp{%
Carl 392
}{%
3
}%
\exmp{%
Carl 391
}{%
3
}%
\exmp{%
Carl 393
}{%
3
}%
\end{example}

\Explanations

In the solution shown in the examples, Alice and Bob agreed as follows:
when $a = 3$, Alice will say $x = 392$ to Carl,
and then, if $391 \le y \le 393$, Bob will know that $a$ was equal to $3$.
The first example shows the first run of such solution when $a = 3$.

The next three examples correspond to the first three tests
in the testing system.
They show the second run with $a = 3$ and different behaviors for Carl.
In the first case, Carl remembers the number clearly without modification,
in the second case, he decreases the number by one,
and in the third case, he increases it by one.

Sure enough, in your solution, Alice and Bob can agree
to do something else entirely, so, generally,
your solution is not required to have these exact answers
in the same circumstances.

\end{problem}
